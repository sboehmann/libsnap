\section{Einleitung}

\subsection{S.N.A.P}
\begin{frame}
\frametitle{S.N.A.P}
Protokollspezifikation der schwedischen Firma HTC.\\
Entwickelt f\"ur die Produktpalette der Hausautomatisierungssysteme rund um das
Power Line Modem \texttt{PLM-24}.
\end{frame}

\subsection{S.N.A.P Features}
\begin{frame}
\frametitle{S.N.A.P Features}
\begin{itemize}
  \item Easy to learn, use and implement
  \item Free and open network protocol
  \item<alert@2> Scaleable binary protocol with small overhead
  \item Up to 16.7 million node addresses
  \item Up to 24 protocol specific flags 
  \item<alert@3> Optional ACK/NAK request
  \item Optional command mode
  \item<alert@4> 8 different error detecting methods
  \item Media independent (power line, RF, TP, IR etc.)
  \item Works with simplex, half-, full- duplex links
  \item<alert@5> Header is scaleable from 3-12 bytes
  \item User specified number of preamble bytes (0-n)
  \item ...
\end{itemize}
\end{frame}

% kate: word-wrap off; encoding utf-8; indent-width 4; tab-width 4; line-numbers on; mixed-indent off; remove-trailing-space-save on; replace-tabs-save on; replace-tabs on; space-indent on;
% vim:set spell et sw=4 ts=4 nowrap cino=l1,cs,U1:
